\documentclass[margin,line]{resume}

\definecolor{sidebarcolor}{RGB}{27, 69, 59}

\usepackage[utf8]{inputenc}
\usepackage[english,russian]{babel}
\usepackage[T1]{fontenc}
\usepackage{fontawesome}

\begin{document}

{\vspace*{-13mm}\sc \large Anton Grishin — Python Developer} \\
\begin{resume}
  \begin{minipage}[t]{0.55\textwidth}
    \section{\mysidestyle Personal\\Information}
    Moscow, Russia \\
    \faGithub  \space
    \href{https://github.com/alchemmist/}{\texttt{alchemmist}} \\
    \faLinkedin \space
    \href{https://www.linkedin.com/in/anton-grishin-6966a8362/}{\texttt{anton-grishin}}
    \\
    \faPaperPlane \space \href{https://t.me/alchemmist}{\texttt{@alchemmist}} \\
    \faPhone \space
    \href{tel:+1234567890}{\color{blue}\texttt{+7(915)067-2638}}  \\
    \faEnvelope \space
    \href{mailto:anton.ingrish@gmail.com}{\color{blue}\texttt{anton.ingrish@gmail.com}}
  \end{minipage}

  \begin{minipage}[H]{0.18\textwidth}
    \begin{textblock}{7}(10.96, 1.30)
      \includegraphics[width=0.255\textwidth]{../images/avatar.png}
    \end{textblock}
  \end{minipage}

  \vspace{-7mm}
  \section{\mysidestyle About Me}
  I am a student and aspiring developer. I have been programming for 4 years, during this time I have contributed to 44 repositories, made 921 commits, and written 176,729 lines of code.
  I \href{https://www.avito.ru/moskva/predlozheniya_uslug/prepodavatel_programmirovaniya_na_python_2556461612}{teach}
  Python, having trained over 30 students. Co-founder and developer at \href{https://ballkit.ru/}{Ballkit}. Previously a professional \href{https://alchemmist.github.io/CV/attachments/sport.pdf}{volleyball player}.

  \section{\mysidestyle Education}
  \href{https://centraluniversity.ru/}{Central University} —
  Mathematics and Computer Science, 2028 \textit{(1st year)}.
  Major: “Development”. Track: “Computer Science Research”.
  % TODO: clarify track

  \section{\mysidestyle Skills}

  \vspace{0.4mm}
  \begin{description}[leftmargin=0pt, itemindent=*, itemsep=0.2pt]
    \item[Python:] \hangindent=9mm \hangafter=1
      \inlinecode{FastAPI}, \inlinecode{pydantic},
      \inlinecode{Flask},
      \inlinecode{SQLAlchemy}, \inlinecode{faststream},
      \inlinecode{alembic}, \inlinecode{pytest}.
    \item[Databases:] \inlinecode{postgres}, \inlinecode{sqlite},
      \inlinecode{redis}, \inlinecode{Yandex Object Storage}.
    \item[Message brokers:] \inlinecode{RabbitMQ}, \inlinecode{Mosquitto}.
    \item[Other technologies:] \inlinecode{SQL}, \inlinecode{Java},
      \inlinecode{JavaScript}, \inlinecode{bash}.
    \item[Dev tools:] \inlinecode{Docker}, \inlinecode{Podman},
      \inlinecode{Make}, \inlinecode{CI/CD}, \inlinecode{Linux},
      \inlinecode{Git}
  \end{description}

  \section{\mysidestyle Achievements}
  Reached the \textbf{final}
  (\href{https://alchemmist.github.io/CV/attachments/russian-chemp-final.pdf}{\texttt{1}})
  of the Russian Championship
  in sports programming, among 5000 participants. Developed a
  \href{https://alchemmist.github.io/CV/attachments/architect.pdf}{microservice architecture}
  of 9 services for the
  \href{https://github.com/alchemmist/sportprog}{web application}
  aggregating sports events across Russia. And developed an
  \href{https://github.com/alchemmist/sport-afisha/blob/main/event_parsing_service/parse_pdf.py}{algorithm}
  for processing and validating annual government reports on sporting events.
  Tech stack: \inlinecode{Kafka}, \inlinecode{React},
  \inlinecode{RabbitMQ (FastStream)},
  \inlinecode{FastAPI},
  \inlinecode{OAuth}.

  \textbf{\href{https://alchemmist.github.io/CV/attachments/scince-for-life-win.pdf}{Winner}
  of the “Science for Life” scientific-practical conference}
  among 117 projects with a smart home system project for private and public
  educational institutions. Tech stack: \inlinecode{Redis},
  \inlinecode{Zigbee2MQTT}, \inlinecode{websockets}, \inlinecode{Go},
  \inlinecode{Python}, \inlinecode{Flask}, \inlinecode{React}.

  \textbf{\href{https://alchemmist.github.io/CV/attachments/informatics-olimpic.pdf}{Third Prize}
  winner}, MPGU Informatics Olympiad:
  “Applied Informatics”

  \section{\mysidestyle Certificates}
  \textbf{Yandex Lyceum}. Entered the 2nd year of the
  \href{https://lyceum.yandex.ru/}{Yandex Academy Lyceum}
  in “Industrial Development” as an external student. Solved 236 problems,
  implemented and defended 3 projects, final scored 100/100. Graduated with
  a \href{https://alchemmist.github.io/CV/attachments/yandex-lyceum.pdf}{diploma with honors}.

  \section{\mysidestyle Personal Projects}\vspace{2mm}

  \begin{description}

    \item[SmartCab]\small{\href{https://github.com/smart-cab}{Smart home system for schools} \hfill
      \textsl{September 2023 — May 2024\vspace{1mm}}}\\
      Developed design \href{https://www.figma.com/design/8H1tFpxgmIDV1xp06ndi73/SmartCab?node-id=0-1&p=f}{prototypes}.
      Implemented them
      (\texttt{\href{https://github.com/smart-cab/smart-cab-hub/tree/main/frontend/src}{1}},
      \texttt{\href{https://github.com/smart-cab/smart-cab-dashboard/tree/main/frontend/src}{2}})
      in \inlinecode{React}.
      \href{https://github.com/smart-cab/smartcab-bot}{Wrote a bot}
      for system administration.
      \href{https://github.com/smart-cab/smartcab-hub/blob/main/backend/smartcab/interface/mqtt.py}{Implemented}
      interaction with 5 smart devices via the Zigbee protocol through
      Mosquitto broker, and 3 scenarios in the classroom.
      Tech: \inlinecode{Python}, \inlinecode{Go},
      \inlinecode{JavaScript}, \inlinecode{bash},
      \inlinecode{Flask},
      \inlinecode{sqlite}, \inlinecode{redis},
      \inlinecode{postgres}, \inlinecode{JsSip}.
      \vspace{3mm}

    \item[Starship]\small{\href{https://github.com/starship-crew}{Text-based action game on Telegram} \hfill
      \textsl{March 2023 — April 2023\vspace{1mm}}}\\
      Developed mechanics for
      \href{https://github.com/starship-crew/telegram-client/blob/main/app/handlers/fight.py}{online duels}
      with real players in a Telegram bot. Also created a flexible
      \href{https://github.com/starship-crew/telegram-client/blob/main/app/template.py}{message rendering engine}
      using jinja2 templates to handle numerous scenario and environment texts.
      Tech: \inlinecode{Python}, \inlinecode{Flask},
      \inlinecode{alembic},
      \inlinecode{python-telegram-bot}, \inlinecode{SQLAlchemy},
      \inlinecode{jinja2}.
      \vspace{3mm}

    \item[Corpse]\small{\href{https://github.com/corpse-inc/corpse}{Game engine} \hfill
      \textsl{November 2022 — January 2023\vspace{1mm}}}\\
      As part of a team, developed a fully functional modular game engine
      based on the \href{https://en.wikipedia.org/wiki/Entity_component_system#:~:text=Entity%E2%80%93component%E2%80%93system%20(ECS,Entity%E2%80%93Component%E2%80%93System%20layout.}{ECS architecture}.
      Also created a survival top-down shooter game set in a zombie apocalypse.
      Tech: \inlinecode{Python}, \inlinecode{Pygame},
      \inlinecode{esper}, \inlinecode{pytmx}, \inlinecode{aseprite},
      \inlinecode{Tiled}.

    \end{description}

    \section{\mysidestyle Custom Development}\vspace{2mm}

    \begin{description}
      \item[Telegram Bots]\small{
          \begingroup
          \textcolor{gray!40}{\textit{All source code publication approved}}
          \endgroup
        }{\hfill
        \textsl{Summer 2024}}\vspace{2mm}
        \begin{list2}
        \item{Developed a
            \href{https://github.com/alchemmist/vpn-bot}{bot} for
            administering and selling WireGuard keys by subscription.
            \href{https://github.com/alchemmist/vpn-bot/blob/develop/vpn_bot/handlers/monthly_analytics.py}{Implemented}
            clear \href{https://github.com/alchemmist/vpn-bot/tree/develop?tab=readme-ov-file#demo}{analytics}
            for clients directly in the bot.
            Tech: \inlinecode{Python}, \inlinecode{sqlite},
            \inlinecode{python-telegram-bot}.
          }
          \vspace{2mm}
        \item{\href{https://github.com/alchemmist/content-parser}{Bot}
            for aggregating relevant messages from chats, channels, and
            groups by keywords.
            Tech:
            \inlinecode{Python},
            \inlinecode{theleton}, \inlinecode{redis}.
          }
          \vspace{2mm}
        \end{list2}
      \item[Other Projects]\small{
          \begingroup
          \textcolor{gray!40}{\textit{All source code publication approved}}
          \endgroup
        }{\hfill
        \textsl{Summer 2024}}\vspace{2mm}
        \begin{list2}

        \item{
            \href{https://github.com/alchemmist/portu-hack}{Parser}
            for the Portuguese embassy with authorization
            (\href{https://github.com/alchemmist/portu-hack/blob/develop/parser/parser/searcher/auth.py}{\texttt{1}}),
            captcha solving
            (\href{https://github.com/alchemmist/portu-hack/blob/develop/parser/parser/searcher/captcha.py}{\texttt{2}}),
            and branching form traversal
            (\href{https://github.com/alchemmist/portu-hack/blob/develop/parser/parser/searcher/snif.py}{\texttt{3}}).
            Created a
            \href{https://github.com/alchemmist/portu-hack/blob/develop/parser/Dockerfile}{Docker image}
            for containerized parsing with \inlinecode{selenium} under
            Chromium. Configured \href{https://github.com/alchemmist/portu-hack/blob/develop/pre-commit-config.yaml}{pre-commit}
            hooks for \inlinecode{ruff} linter and \inlinecode{pyright} static analyzer.
            Tech: \inlinecode{Python}, \inlinecode{selenium},
            \inlinecode{psycopg}, \inlinecode{SQLAlchemy},
            \inlinecode{FastAPI}.}
          \vspace{2mm}

        \item{
            \href{https://github.com/alchemmist/outline-vpn-cli}{CLI tool}
            for secure operations on Outline servers for thousands of clients.
            Tech: \inlinecode{Python}.}
        \end{list2}

    \end{description}

    \vspace{-4mm}
    \section{\mysidestyle Interests}\vspace{0.7mm}

    {\textbf{Formal Verification:} I am taking the
      \href{https://softwarefoundations.cis.upenn.edu}{Software Foundations}
      course. I have read the first volume and
      \href{https://github.com/alchemmist/coq-learning}{proved} 250
    theorems in \inlinecode{Coq}.} \\

    \vspace{-6mm}

    \textbf{Linux:} I use Arch with the Hyprland compositor. I have published
    my \href{https://github.com/alchemmist/.dotfiles}{\texttt{.dotfiles}}.
    I wrote my Neovim
    \href{https://github.com/alchemmist/.dotfiles/tree/main/nvim}{config}
    and a color
    \href{https://github.com/alchemmist/nothing.nvim}{theme} from scratch.
    I have written over 20 custom
    \href{https://github.com/alchemmist/.dotfiles/tree/main/scripts}{scripts}.

  \end{resume}

  \begin{minipage}[H]{9.18\textwidth}
    \begin{textblock}{7}(-0.65, 15.1)
      \begingroup
      \hspace{35mm}
      \hypersetup{urlcolor=gray!90}
      \large
      \href{https://github.com/alchemmist}{→ more projects on
      \underline{GitHub}}
      \endgroup
    \end{textblock}

  \end{minipage}

  \clearpage

  \end{document}

