\documentclass[margin,line]{resume}

\usepackage[utf8]{inputenc}
\usepackage[english,russian]{babel}
\usepackage[T1]{fontenc}
\usepackage{fontawesome}

% TODO: добавить схему микросервсиной архитектуры с хакатона как пруф
% TODO: сделать автовыраваниване для фотки
% TODO: возможно весь технический скилл написать?
% TODO: сплющить проекты с одинаковым стеком технологий
% TODO: про стартап нписать, поговорить с Марком чтоб его заопенсорсить

\begin{document}

{\vspace*{-13mm}\sc \large Anton Grishin --- Backend Developer} \\
\begin{resume}
  \begin{minipage}[t]{0.55\textwidth}
    \section{\mysidestyle Personal\\Information}
    Anton Grishin \\
    Moscow, Russia \\
    \faGithub  \space
    \href{https://github.com/alchemmist/}{\texttt{alchemmist}} \\
    \faLinkedin \space
    \href{https://www.linkedin.com/in/anton-grishin-6966a8362/}{\texttt{anton-grishin}}
    \\
    \faPaperPlane \space \href{https://t.me/alchemmist}{\texttt{@alchemmist}} \\
    \faPhone \space
    \href{tel:+1234567890}{\color{blue}\texttt{+7(915)067-2638}}  \\
    \faEnvelope \space
    \href{mailto:anton.ingrish@gmail.com}{\color{blue}\texttt{anton.ingrish@gmail.com}}
  \end{minipage}

  \begin{minipage}[H]{0.18\textwidth}
    \begin{textblock}{7}(10.84, 1.37)
      \includegraphics[width=0.27\textwidth]{../images/avatar.png}
    \end{textblock}
  \end{minipage}

  \vspace{-7mm}
  \section{\mysidestyle About Me}
  I am a student and an aspiring developer. I have been programming
  for 4 years, during which I contributed to 44 repositories, made
  921 commits, and wrote 176,729 lines of code. Former professional
  \href{https://alchemmist.github.io/CV/attachments/sport.pdf}{volleyball
  player}. I am also a
  \href{https://www.avito.ru/moskva/predlozheniya_uslug/prepodavatel_programmirovaniya_na_python_2556461612}{Python
  instructor}, having taught over 30 students. Co-founder and
  developer at \href{https://ballkit.ru/}{Ballkit}.

  \section{\mysidestyle Skills}

  \vspace{0.5mm}

  \begin{description}[leftmargin=0pt, itemindent=*]
    \item[Python:] \hangindent=9mm \hangafter=1
      \inlinecode{FastAPI}, \inlinecode{pydantic},
      \inlinecode{Flask},
      \inlinecode{SQLAlchemy}, \inlinecode{faststream},
      \inlinecode{alembic}, \inlinecode{pytest}.
    \item[Go:] \inlinecode{http}, \inlinecode{grpc},
      \inlinecode{protobuf}, \inlinecode{tgbotapi},
      \inlinecode{reflect}, \inlinecode{gofsm}.
    \item[Databases:] \inlinecode{postgres}, \inlinecode{sqlite},
      \inlinecode{redis}, \inlinecode{Yandex Object Storage}.
    \item[Message brokers:] \inlinecode{RabbitMQ}, \inlinecode{Mosquitto}.
    \item[Other technologies:] \inlinecode{SQL}, \inlinecode{Java},
      \inlinecode{JavaScript}, \inlinecode{bash}.
    \item[Dev tools:] \inlinecode{Docker}, \inlinecode{Podman},
      \inlinecode{Make}, \inlinecode{CI/CD}, \inlinecode{Linux},
      \inlinecode{Git}
  \end{description}

  \section{\mysidestyle Education}
  \href{https://centraluniversity.ru/}{Central University} ---
  Mathematics and Computer Science, 2028

  \section{\mysidestyle Certifications}
  \textbf{Yandex Lyceum}. Admitted externally to the 2nd year of the
  \href{https://lyceum.yandex.ru/}{Yandex Academy Lyceum} in
  "Industrial Development." Solved 236 problems, implemented and
  defended 3 projects, scored 100/100 on the final. Graduated with
  \href{https://alchemmist.github.io/CV/attachments/yandex-lyceum.pdf}{honors}.

  \vspace{-6mm}

  \hfill \textsl{September 2022 — April 2023}

  \textbf{Python Course}. Completed the
  \href{https://stepik.org/course/67}{Stepik course} on
  \inlinecode{Python}, solving 71 exercises and finishing with
  \href{https://alchemmist.github.io/CV/attachments/stepik-python-course.pdf}{distinction}.

  \vspace{-7mm}

  \hfill \textsl{September 2021 — April 2022}

  \section{\mysidestyle Achievements}
  Finalist of the \textbf{Russian Championship} in competitive
  programming (among 5000 participants). Developed a
  \href{https://alchemmist.github.io/CV/attachments/architect.pdf}{microservice
  architecture} of 9 services for a
  \href{https://github.com/alchemmist/sportprog}{web application}
  aggregating sports events across Russia. Also created an
  \href{https://github.com/alchemmist/sport-afisha/blob/main/event_parsing_service/parse_pdf.py}{algorithm}
  for processing and validating annual state reports on sports
  events. Tech stack: \inlinecode{Kafka}, \inlinecode{React},
  \inlinecode{RabbitMQ (FastStream)}, \inlinecode{FastAPI}, \inlinecode{OAuth}.

  \vspace{-6mm}

  \hfill \textsl{December 2024}

  \textbf{Winner} of the "Science for Life" scientific-practical
  conference among 117 projects with a smart home project for private
  and public educational institutions. Tech stack:
  \inlinecode{Redis}, \inlinecode{Zigbee2MQTT},
  \inlinecode{websockets}, \inlinecode{Go}, \inlinecode{Python},
  \inlinecode{Flask}, \inlinecode{React}.
  \vspace{-6mm}

  \hfill \textsl{April 2024}

  Participant in the \textbf{Nuclear IT Hack} hackathon, where our
  team worked for 48 hours on a Rosatom case: a service for detecting
  emotional tone in online meetings.

  \vspace{-7mm}

  \hfill \textsl{April 2024}

  \textbf{3rd Prize} in the MPGU Olympiad in Informatics: "Applied Informatics".
  \vspace{-2mm}

  \hfill \textsl{June 2023}
  \vspace{-4mm}
  \section{\mysidestyle Interests}\vspace{0.7mm}

  {\textbf{Formal Verification:} Currently taking the
    \href{https://softwarefoundations.cis.upenn.edu}{Software
    Foundations} course. Completed the first volume and
    \href{https://github.com/alchemmist/coq-learning}{proved} 250
  theorems in \inlinecode{Coq}.} \\

  \vspace{-6mm}

  \textbf{Linux:} Using Arch with the Hyprland compositor. Published
  my \href{https:/github.com/alchemmist/.dotfiles}{.dotfiles}. Wrote
  my Neovim
  \href{https://github.com/alchemmist/.dotfiles/tree/main/nvim}{config}
  from scratch and a color
  \href{https://github.com/alchemmist/nothing.nvim}{theme}. Authored
  598 lines of custom
  \href{https://github.com/alchemmist/.dotfiles/tree/main/scripts}{scripts}.

  \vfill
  % ------------------------------------------------------------------ %

  \section{\mysidestyle Startup Experience}\vspace{2mm}

  \begin{description}

    \item[Patch Loyalty]\small Loyalty program for SMBs \hfill
      \textsl{July 2024 — August 2024}\vspace{1mm}\\
      Developed a Single Message Bot in Telegram using
      \inlinecode{Go} for the boxed loyalty solution
      \href{https://ballkit.ru}{\texttt{ballkit.ru}}.

      \textbf{Technologies:}
      \inlinecode{Go},
      \inlinecode{tgbotapi}, \inlinecode{grpc},
      \inlinecode{gofsm}.
  \end{description}

  \section{\mysidestyle Personal Projects}\vspace{2mm}

  \begin{description}

    \item[SmartCab]\small{\href{https://github.com/smart-cab}{School
        smart home system} \hfill
      \textsl{September 2023 — May 2024}\vspace{1mm}}\\
      Designed
      \href{https://www.figma.com/design/8H1tFpxgmIDV1xp06ndi73/SmartCab?node-id=0-1&p=f}{admin
      panel and tablet interface prototypes} and implemented them in
      \inlinecode{React}. Developed a bot for system administration
      and implemented interaction with five smart devices over Zigbee
      via Mosquitto broker and three classroom scenarios.
      Technologies: \inlinecode{Python}, \inlinecode{Go},
      \inlinecode{JavaScript}, \inlinecode{bash}, \inlinecode{Flask},
      \inlinecode{sqlite}, \inlinecode{redis}, \inlinecode{postgres},
      \inlinecode{JsSip}.
      \vspace{3mm}

    \item[Starship]\small{\href{https://github.com/starship-crew}{Text-based
        action game in Telegram} \hfill \textsl{March 2023 — April
      2023}\vspace{1mm}}\\
      Developed real-time duel mechanics for players in a Telegram
      bot and a flexible message rendering engine using jinja2
      templates for complex narrative texts. Technologies:
      \inlinecode{Python}, \inlinecode{Flask}, \inlinecode{alembic},
      \inlinecode{python-telegram-bot}, \inlinecode{SQLAlchemy},
      \inlinecode{jinja2}.

      \vspace{3mm}

    \item[Corpse]\small{\href{https://github.com/corpse-inc/corpse}{Game
      engine} \hfill \textsl{November 2022 — January 2023}\vspace{1mm}}\\
      Collaborated on a modular ECS-based game engine and built a
      survival top-down shooter about a zombie apocalypse.
      Technologies: \inlinecode{Python}, \inlinecode{Pygame},
      \inlinecode{esper}, \inlinecode{pytmx}, \inlinecode{aseprite},
      \inlinecode{Tiled}.

      \vspace{3mm}

    \item[SchoolRing]\small{\href{https://github.com/alchemmist/school-ring}{School
        bell scheduling} \begingroup
        \hypersetup{urlcolor=blue!30}
        \href{https://github.com/}{\texttt{release v0.0.2}}
        \endgroup
        \hfill
      \textsl{October 2022 — November 2022}\vspace{1mm}}\\
      Developed a desktop application in \inlinecode{PyQt6}.
      Configured \href{https://github.com}{\texttt{pre-commit}} hooks
      to auto-convert .ui files to .py and CI actions for
      cross-platform releases. Recently migrated the project to
      \href{https://astral.sh/blog/uv}{\texttt{uv}} for dependency
      management. Technologies: \inlinecode{Python},
      \inlinecode{PyQt6}, \inlinecode{threading}, \inlinecode{schedule}.

  \end{description}

  \section{\mysidestyle Contract Development}\vspace{2mm}

  \begin{description}
    \item[Telegram Bots]\small{
        \begingroup
        \textcolor{gray!40}{\textit{Publication of all sources approved}}
        \endgroup
      }{\hfill
      \textsl{Summer 2024}}\vspace{2mm}
      \begin{list2}
      \item{Developed a
          \href{https://github.com/alchemmist/vpn-bot}{bot} for
          administering and selling WireGuard keys via subscription.
          Implemented inline analytics
          \href{https://github.com/alchemmist/vpn-bot/tree/develop?tab=readme-ov-file#demo}{reports}
          directly in the bot. Technologies: \inlinecode{Python},
          \inlinecode{sqlite}, \inlinecode{python-telegram-bot}.
        }
        \vspace{2mm}
      \item{\href{https://github.com/alchemmist/content-parser}{Bot}
          for aggregating relevant messages from chats, channels, and
          groups by keywords. Technologies: \inlinecode{Python},
          \inlinecode{theleton}, \inlinecode{redis}
        }

        \vspace{2mm}

      \end{list2}
    \item[Other Projects]\small{
        \begingroup
        \textcolor{gray!40}{\textit{Publication of all sources approved}}
        \endgroup
      }{\hfill
      \textsl{Summer 2024}}\vspace{2mm}
      \begin{list2}

      \item{\href{https://github.com/alchemmist/portu-hack}{Parser}
          for the Portuguese embassy website with authentication
          \texttt{auth.py}, captcha solving \texttt{captcha.py}, and
          branching form traversal \texttt{snif.py}. Created a Docker
          image for containerized parsing using Selenium on Chromium.
          Configured \texttt{pre-commit} hooks for Ruff linter and
          Pyright static analysis. Technologies: \inlinecode{Python},
          \inlinecode{selenium}, \inlinecode{psycopg},
        \inlinecode{SQLAlchemy}, \inlinecode{FastAPI}.}

        \vspace{2mm}

      \item{\href{https://github.com/alchemmist/outline-vpn-cli}{CLI
          tool} for secure operations on Outline servers across
        thousands of clients. Technologies: \inlinecode{Python}.}
      \end{list2}

  \end{description}

  \section{\mysidestyle Lab Works}\vspace{2mm}

  \begin{description}

    \item[BookSwap]\small{\href{https://github.com/alchemmist/bookswap}{Book
        exchange service} \hfill
      \textsl{May 2024}\vspace{1mm}}\\
      Designed the database schema
      \href{https://github.com/alchemmist/bookswap/blob/dev/backend/src/main/resources/schema.sql}{definition}.
      Implemented the backend with core business logic in
      \inlinecode{Spring} and the
      \href{https://github.com/alchemmist/bookswap/tree/dev/auth-service}{authentication
      service} in \inlinecode{Go}, plus a
      \href{https://github.com/alchemmist/bookswap/tree/dev/frontend}{frontend}
      in \inlinecode{React}.

      \textbf{Technologies:} \inlinecode{Java},
      \inlinecode{SpringBoot}, \inlinecode{Lombok},
      \inlinecode{JavaScript}, \inlinecode{React}, \inlinecode{Go},
      \inlinecode{crypto}, \inlinecode{pgx}.

  \end{description}
\end{resume}

\begin{minipage}[H]{9.18\textwidth}
  \begin{textblock}{7}(-0.65, 15.1)
    \begingroup
    \hspace{35mm}
    \hypersetup{urlcolor=gray!90}
    \large
    \href{https://github.com/alchemmist}{→ more projects on \underline{GitHub}}
    \endgroup
  \end{textblock}

\end{minipage}

\clearpage

\end{document}
