\documentclass[margin,line]{resume}

\definecolor{sidebarcolor}{RGB}{27, 69, 59}

\usepackage[utf8]{inputenc}
\usepackage[english,russian]{babel}
\usepackage[T1]{fontenc}
\usepackage{fontawesome}

\begin{document}

{\vspace*{-13mm}\sc \large Гришин Антон — Python разработчик} \\
\begin{resume}
  \begin{minipage}[t]{0.55\textwidth}
    \section{\mysidestyle Персональная\\Информация}
    Москва, Россия \\
    \faHome  \space
    \href{https://alchemmist.xyz}{\texttt{alchemmist.xyz}} \\
    \faGithub  \space
    \href{https://github.com/alchemmist/}{\texttt{alchemmist}} \\
    \faLinkedin \space
    \href{https://www.linkedin.com/in/alchemmist/}{\texttt{alchemmist}}
    \\
    \faPaperPlane \space \href{https://t.me/alchemmist}{\texttt{@alchemmist}} \\
    \faPhone \space
    \href{tel:+1234567890}{\color{blue}\texttt{+7(915)067-2638}}  \\
    \faEnvelope \space
    \href{mailto:anton.ingrish@gmail.com}{\color{blue}\texttt{anton.ingrish@gmail.com}}
  \end{minipage}

  \begin{minipage}[H]{0.18\textwidth}
    % 10.87 1.38
    \begin{textblock}{7}(10.87, 1.38)
      \includegraphics[width=0.270\textwidth]{../images/avatar.png}
    \end{textblock}
  \end{minipage}


  \vspace{-7mm}
  \section{\mysidestyle Обо мне}
  Я студент, начинающий разработчик. Программирую 4 года, за
  это время участвовал в разработке 44 репозиториев, отправил 921
  коммит и написал 176729 строчек кода.
  \href{https://www.avito.ru/moskva/predlozheniya_uslug/prepodavatel_programmirovaniya_na_python_2556461612}{Преподаю}
  Python, обучил $> 30$ учеников. Сооснователь и
  разработчик в \href{https://ballkit.ru/}{Баллкит}. В прошлом профессиональный
  \href{https://alchemmist.github.io/CV/attachments/sport.pdf}{волейболист}.

  \section{\mysidestyle Образование}
  \href{https://centraluniversity.ru/}{Центральный Университет} —
  Математика и компьютерные науки, 2028 \textit{(1 курс)}.
  Направление «Разработка». Траектория «Computer Science Research».
  % TODO: уточнить трек

  \section{\mysidestyle Навыки}

  \vspace{0.4mm}
  \begin{description}[leftmargin=0pt, itemindent=*, itemsep=0.2pt]
    \item[Python:] \hangindent=9mm \hangafter=1
      \inlinecode{FastAPI}, \inlinecode{pydantic},
      \inlinecode{Flask},
      \inlinecode{SQLAlchemy}, \inlinecode{faststream},
      \inlinecode{alembic}, \inlinecode{pytest}.
    \item[Databases:] \inlinecode{postgres}, \inlinecode{sqlite},
      \inlinecode{redis}, \inlinecode{Yandex Object Storage}.
    \item[Message brokers:] \inlinecode{RabbitMQ}, \inlinecode{Mosquitto}.
    \item[Other technologies:] \inlinecode{SQL}, \inlinecode{Java},
      \inlinecode{JavaScript}, \inlinecode{bash}.
    \item[Dev tools:] \inlinecode{Docker}, \inlinecode{Podman},
      \inlinecode{Make}, \inlinecode{CI/CD}, \inlinecode{Linux},
      \inlinecode{Git}
  \end{description}

  \section{\mysidestyle Достижения}
  Прошёл в
  \textbf{финал}
  (\href{https://alchemmist.github.io/CV/attachments/russian-chemp-final.pdf}{\texttt{1}})
  \href{https://events.fsp-russia.com/championship}{Чемпионата
  России} по спортивному
  программированию, среди 5000 участников. Разработал
  \href{https://alchemmist.github.io/CV/attachments/architect.pdf}{микросервисную
  архитектуру} из 9 сервисов для
  \href{https://github.com/alchemmist/sportprog}{веб-приложения},
  агрегирующего спортивные события по
  всей России. И разработал
  \href{https://github.com/alchemmist/sport-afisha/blob/main/event_parsing_service/parse_pdf.py}{алгоритм}
  обработки и проверки ежегодных
  государственных
  отчётов о спортивных мероприятиях. Технический стек:
  \inlinecode{Kafka}, \inlinecode{React},
  \inlinecode{RabbitMQ (FastStream)},
  \inlinecode{FastAPI},
  \inlinecode{OAuth}.

  \textbf{\href{https://alchemmist.github.io/CV/attachments/scince-for-life-win.pdf}{Победитель}
  научно-практической конференции}
  «\href{https://conf.profil.mos.ru/academ}{Наука для
  жизни}» среди 117 проектов с
  \href{https://github.com/smart-cab/}{проектом умного
  дома} для частных и государственных
  образовательных учреждений. Технический стек: \inlinecode{Redis},
  \inlinecode{Zigbee2MQTT}, \inlinecode{websockets}, \inlinecode{Go},
  \inlinecode{Python}, \inlinecode{Flask}, \inlinecode{React}.

  \textbf{\href{https://alchemmist.github.io/CV/attachments/informatics-olimpic.pdf}{Призёр}
  3-й степени}, олимпиады МПГУ по информатике:
  «Прикладная информатика»

  \section{\mysidestyle Сертификаты}
  \textbf{Яндекс Лицей}. Экстерном поступил на 2-й курс
  \href{https://lyceum.yandex.ru/}{Лицея
  Академии Яндекса} по «Промышленной Разработке». Решил 236 задач,
  реализовал и защитил 3 проекта, финальный на 100/100 баллов. Окончил с
  \href{https://alchemmist.github.io/CV/attachments/yandex-lyceum.pdf}{аттестатом
  с отличием}.

  \section{\mysidestyle Петпроекты}\vspace{2mm}

  \begin{description}

    \item[SmartCab]\small{\href{https://github.com/smart-cab}{Система
        умного дома в школе} \hfill
      \textsl{Сентябрь 2023 — Май 2024\vspace{1mm}}}\\
      Разработал
      \href{https://www.figma.com/design/8H1tFpxgmIDV1xp06ndi73/SmartCab?node-id=0-1&p=f}{прототипы}
      дизайна. Сверстал их
      (\texttt{\href{https://github.com/smart-cab/smartcab-hub/tree/main/frontend/src}{1}},
      \texttt{\href{https://github.com/smart-cab/smartcab-dashboard/tree/main/frontend/src}{2}})
      на \inlinecode{React}.
      \href{https://github.com/smart-cab/smartcab-bot}{Написал бота}
      для администрирования системы.
      \href{https://github.com/smart-cab/smartcab-hub/blob/main/backend/smartcab/interface/mqtt.py}{Реализовал}
      взаимодействие с 5-ю умными устройствами по протокулу Zigbee, через
      брокер сообщений Mosquitto, и 3 сценария в кабинете.
      Технологии:
      \inlinecode{Python}, \inlinecode{Go},
      \inlinecode{JavaScript}, \inlinecode{bash},
      \inlinecode{Flask},
      \inlinecode{sqlite}, \inlinecode{redis},
      \inlinecode{postgres}, \inlinecode{JsSip}.
      \vspace{3mm}

    \item[Starship]\small{\href{https://github.com/starship-crew}{Текстовая
        экшн игра в Telegram}
      \hfill \textsl{Март 2023 — Апрель 2023\vspace{1mm}}}\\
      Разработал механику
      \href{https://github.com/starship-crew/telegram-client/blob/main/app/handlers/fight.py}{онлайн
      дуэлей} с реальными игроками в Telegram боте. А
      так же удобный
      \href{https://github.com/starship-crew/telegram-client/blob/main/app/template.py}{движок
      рендеринга}
      сообщений в чате через jinja2 шаблоны, так как для текстовой игры
      требовалось гибко работать с большим количеством сценарных и
      пейзажных текстов.
      Технологии:
      \inlinecode{Python}, \inlinecode{Flask},
      \inlinecode{alembic},
      \inlinecode{python-telegram-bot}, \inlinecode{SQLAlchemy},
      \inlinecode{jinja2}.

      \vspace{3mm}

    \item[Corpse]\small{\href{https://github.com/corpse-inc/corpse}{Игровой
        движок}
      \hfill \textsl{Ноябрь 2022 — Январь 2023\vspace{1mm}}}\\
      В команде разработали полнофункциональный модульный
      игровой движок с архитектурой
      \href{https://en.wikipedia.org/wiki/Entity_component_system#:~:text=Entity%E2%80%93component%E2%80%93system%20(ECS,Entity%E2%80%93Component%E2%80%93System%20layout.}{ECS}.
        А также survival top-down шутер игру о зомби-апокалипсисе.
        Технологии:
        \inlinecode{Python}, \inlinecode{Pygame},
        \inlinecode{esper}, \inlinecode{pytmx}, \inlinecode{aseprite},
        \inlinecode{Tiled}.

    \end{description}

    \section{\mysidestyle Заказная разработка}\vspace{2mm}

    \begin{description}
      \item[Telegram боты]\small{
          \begingroup
          \textcolor{gray!40}{\textit{Публикация всех исходников согласована}}
          \endgroup
        }{\hfill
        \textsl{Лето 2024}}\vspace{2mm}
        \begin{list2}
        \item{Разработал
            \href{https://github.com/alchemmist/vpn-bot}{бота} для
            администрирования и продажи WireGuard
            ключей по подписке.
            \href{https://github.com/alchemmist/vpn-bot/blob/develop/vpn_bot/handlers/monthly_analytics.py}{Написал}
            наглядную
            \href{https://github.com/alchemmist/vpn-bot/tree/develop?tab=readme-ov-file#demo}{аналитику}
            по клиентам прямо в боте.
            Технологии: \inlinecode{Python}, \inlinecode{sqlite},
            \inlinecode{python-telegram-bot}.
          }
          \vspace{2mm}
        \item{\href{https://github.com/alchemmist/content-parser}{Бот}
            для агрегирования релевантных сообщений из чатов, каналов и
            групп по ключевым словам.
            Технологии:
            \inlinecode{Python},
            \inlinecode{theleton}, \inlinecode{redis}
          }

          \vspace{2mm}

        \end{list2}
      \item[Другие проекты]\small{
          \begingroup
          \textcolor{gray!40}{\textit{Публикация всех исходников согласована}}
          \endgroup
        }{\hfill
        \textsl{Лето 2024}}\vspace{2mm}
        \begin{list2}

        \item{
            \href{https://github.com/alchemmist/portu-hack}{Парсер}
            португальского посольства с механизмами авторизации
            (\href{https://github.com/alchemmist/portu-hack/blob/develop/parser/parser/searcher/auth.py}{\texttt{1}}),
            разрешения капчи
            (\href{https://github.com/alchemmist/portu-hack/blob/develop/parser/parser/searcher/captcha.py}{\texttt{2}}),
            и
            ветвящимся обходом форм на сайте
            (\href{https://github.com/alchemmist/portu-hack/blob/develop/parser/parser/searcher/snif.py}{\texttt{3}}).
            Написал
            \href{https://github.com/alchemmist/portu-hack/blob/develop/parser/Dockerfile}{Docker
            образ} для контейнеризованного парсинга
            \inlinecode{selenium} из под движка \texttt{Chromium}.
            Настроил
            \href{https://github.com/alchemmist/portu-hack/blob/develop/pre-commit-config.yaml}{\texttt{pre-commit}}
            хуки для линтера
            \inlinecode{ruff} и статического анализатора \inlinecode{pyright}.
            Технологии: \inlinecode{Python}, \inlinecode{selenium},
            \inlinecode{psycopg},
            \inlinecode{SQLALchemy},
          \inlinecode{FastAPI}.}

          \vspace{2mm}

        \item{
            % \begingroup
            % \hypersetup{urlcolor=blue!30}
            % \href{https://github.com/}{\texttt{release v0.0.1}}
            % \endgroup
            % TODO: добавить релиз ?? если надо
            \href{https://github.com/alchemmist/outline-vpn-cli}{Cli-инструмент}
            для безопасного проведения операций на
            Outline серверах на тысячах клиентов.
          Технологии: \inlinecode{Python}.}
        \end{list2}

    \end{description}

    \vspace{-4mm}
    \section{\mysidestyle Интересы}\vspace{0.7mm}

    {\textbf{Формальная верификация:} Прохожу курс
      \texttt{\href{https://softwarefoundations.cis.upenn.edu}{softwarefoundations}}.
      Прочитал первый том,
      \href{https://github.com/alchemmist/coq-learning}{доказал} 250
    теорем на \inlinecode{Coq}.} \\

    \vspace{-6mm}

    \textbf{Linux:} Использую Arch с композитором Hyprland. Опубликовал
    свои
    \href{https:/github.com/alchemmist/.dotfiles}{\texttt{.dotfiles}}.
    С нуля написал свой Neovim
    \href{https://github.com/alchemmist/.dotfiles/tree/main/nvim}{конфиг}
    и цветовую
    \href{https://github.com/alchemmist/nothing.nvim}{тему}.
    Написал более 20 кастомных
    \href{https://github.com/alchemmist/.dotfiles/tree/main/scripts}{скриптов}.

  \end{resume}

  \begin{minipage}[H]{9.18\textwidth}
    \begin{textblock}{7}(-0.65, 15.1)
      \begingroup
      \hspace{35mm}
      \hypersetup{urlcolor=gray!90}
      \large
      \href{https://github.com/alchemmist}{→ больше проектов на
      \underline{GitHub}}
      \endgroup
    \end{textblock}

  \end{minipage}

  \clearpage

  \end{document}
